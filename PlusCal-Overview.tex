%%%%%%%%%%%%%%%%%%%%%%%%%%%%%%%%%%%%%%%%%%%%%%%%%%%%%%%%%%%%%%%%%%%%%%%%%%%%%%%%
% This presentation uses the Metropolis BEAMER theme:
%   https://github.com/matze/mtheme
%
% The theme requires you install the Mozilla's Fira Sans fonts and XeTex
% to enjoy the typography.
%
% The source code is included using the Minted package that requires
% that Python's pygmentize package is available in your system and
% compile with the -shell-escape option.
%%%%%%%%%%%%%%%%%%%%%%%%%%%%%%%%%%%%%%%%%%%%%%%%%%%%%%%%%%%%%%%%%%%%%%%%%%%%%%%%

\documentclass[10pt]{beamer}

\usetheme{metropolis}
\usepackage{appendixnumberbeamer}

\usepackage{booktabs}
\usepackage[scale=2]{ccicons}

\usepackage{pgfplots}
\usepgfplotslibrary{dateplot}

\usepackage{xspace}
\usepackage{minted}
\usepackage{pifont}

\definecolor{bgCode}{rgb}{0.95,0.95,0.95}

% Custom commands
\newcommand{\tlaplus}{\textbf{\textsc{TLA\textsuperscript{+}}}\xspace}
\newcommand{\pluscal}{\textbf{\textsc{PlusCal}}\xspace}

\title{A \tlaplus and \pluscal Overview}
\subtitle{Some koans and exercises}
\date{\today}
\author{Rafael Luque}
\institute{OSOCO}

\titlegraphic{\hfill\includegraphics[height=1.5cm]{figures/logo-15th-black.pdf}}

\begin{document}

\maketitle

% ToC
\begin{frame}{Table of contents}
  \setbeamertemplate{section in toc}[sections numbered]
  \tableofcontents[hideallsubsections]
\end{frame}

\section{Introduction}

\section{Test the Water}

\section{\pluscal Semantics}

\begin{frame}{About \pluscal}
  \begin{itemize}
    \item Developed by Lamport in 2009 to make \tlaplus more accessible to programmers.
    \item \pluscal provides a pseucode-like structure on top of \tlaplus.
    \item Adds additional syntax: \alert{assignments} and \alert{labels}.
    \item \pluscal is language that compiles down to \tlaplus.
  \end{itemize}
\end{frame}

\subsection{Spec layout}

\defverbatim[colored]\layoutSample{
  \inputminted[bgcolor=bgCode,escapeinside=//]{c}{samples/layout.tla}
}
\begin{frame}{Spec layout}
  \layoutSample
\end{frame}

\begin{frame}{Spec layout}
  \begin{dingautolist}{182}
  \item \tlaplus specs must start with at least four - at each side of \mintinline{c}{MODULE} and four = at the end.
  \item The module name must be the same as filename.
  \item \mintinline{c}{EXTENDS} is the import keyword.
  \item \mintinline{c}{(*...*)} is the block comment. \pluscal spec starts with \mintinline{c}{--algorithm <name>} and ends with \mintinline{c}{end algorithm}.
  \item Initialization of variables.
    \item Where the algorithm is implemented.
  \end{dingautolist}
\end{frame}

\subsection{Values and Standard Operations}

\begin{frame}{Basic values}
  Four basic values in \tlaplus:
  \begin{description}
    \item[String] Must be written en double quotes.
    \item[Integer] Floats are not supported.
    \item[Boolean] Written as TRUE and FALSE.
    \item[Model value] A kind of symbol value.
  \end{description}
\end{frame}

\begin{frame}{Standard operations}
  \begin{table}
    \begin{tabular}{@{} p{2cm}lp{3cm} @{}}
      \toprule
      Operator & Meaning & Example\\
      \midrule
      \mintinline{c}{x = y} & Equals & \mintinline{c}{>> 1 = 2} \newline FALSE \\
      \mintinline{c}{x /= y} & Not Equals & \mintinline{c}{>> 1 /= 2} \newline TRUE\\
      \mintinline{c}{x /\ y} & And & \mintinline{c}{>> TRUE /\ FALSE} \newline FALSE\\
      \mintinline{c}{x \/ y} & Or & \mintinline{c}{>> TRUE \/ FALSE} \newline TRUE\\
      \mintinline{c}{~x} & Not & \mintinline{c}{>> ~TRUE} \newline FALSE\\      
      \mintinline{c}{x := y} & Assignment & PlusCal only\\
      \bottomrule
    \end{tabular}
  \end{table}
\end{frame}

\subsection{Arithmetic Operations}

\begin{frame}{Arithmetic Operations}
  \begin{itemize}
    \item  If you \mintinline{c}{EXTENDS Integers} get the arithmetic operators: \textbf{+}, \textbf{-}, \textbf{\%} and \textbf{*}.
    \item  Decimal division is not supported, only the integer division: \textbf{\mintinline{c}{\div}}.
    \item You also get the range operator \textbf{\mintinline{c}{..}} where \mintinline{c}{a..b} is the set \mintinline{c}{{a, a+1, ..., b-1, b}}.
  \end{itemize}
\end{frame}

\subsection{Sets}

\subsection{Tuples or Sequences}

\subsection{Structs}


\section{\pluscal Koans}

\section{Workshop of Spec Writing}


\begin{frame}[standout]
  Questions?
\end{frame}

\appendix

\begin{frame}[allowframebreaks]{References}

  \nocite{*}  \bibliography{PlusCal}xuxuxux1u
  \bibliographystyle{alpha}

\end{frame}

\end{document}
