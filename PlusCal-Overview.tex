%%%%%%%%%%%%%%%%%%%%%%%%%%%%%%%%%%%%%%%%%%%%%%%%%%%%%%%%%%%%%%%%%%%%%%%%%%%%%%%%
% This presentation uses the Metropolis BEAMER theme:
%   https://github.com/matze/mtheme
%
% The theme requires you install the Mozilla's Fira Sans fonts and XeTex
% to enjoy the typography.
%
% The source code is included using the Minted package that requires
% that Python's pygmentize package is available in your system and
% compile with the -shell-escape option.
%%%%%%%%%%%%%%%%%%%%%%%%%%%%%%%%%%%%%%%%%%%%%%%%%%%%%%%%%%%%%%%%%%%%%%%%%%%%%%%%

\documentclass[10pt]{beamer}

\usetheme{metropolis}
\usepackage{appendixnumberbeamer}
\metroset{block=fill}

\usepackage{booktabs}
\usepackage[scale=2]{ccicons}

\usepackage{pgfplots}
\usepgfplotslibrary{dateplot}

\usepackage{xspace}
\usepackage{minted}
\usepackage{pifont}
\usepackage[font=scriptsize]{caption}

\definecolor{bgCode}{rgb}{0.95,0.95,0.95}

% Custom commands
\newcommand{\tlaplus}{\textbf{\textsc{TLA\textsuperscript{+}}}\xspace}
\newcommand{\pluscal}{\textbf{\textsc{PlusCal}}\xspace}

\title{A \tlaplus and \pluscal Overview}
\subtitle{Some koans and exercises}
\date{\today}
\author{Rafael Luque}
\institute{OSOCO}

\titlegraphic{\hfill\includegraphics[height=1.5cm]{figures/logo-15th-black.pdf}}

\begin{document}

\maketitle

% ToC
\begin{frame}{Table of contents}
  \setbeamertemplate{section in toc}[sections numbered]
  \tableofcontents[hideallsubsections]
\end{frame}

\section{Introduction}

\section{\tlaplus and \pluscal Semantics}

\begin{frame}{About \pluscal}
  \begin{itemize}
    \item Developed by Lamport in 2009 to make \tlaplus more accessible to programmers.
    \item \pluscal provides a pseucode-like structure on top of \tlaplus.
    \item Adds additional syntax: \alert{assignments} and \alert{labels}.
    \item \pluscal is language that compiles down to \tlaplus.
  \end{itemize}
\end{frame}

\subsection{Spec layout}

\defverbatim[colored]\layoutSample{
  \inputminted[bgcolor=bgCode,escapeinside=//]{text}{samples/layout.tla}
}
\begin{frame}{Spec layout}
  \layoutSample
\end{frame}

\begin{frame}{Spec layout}
  \begin{dingautolist}{182}
  \item \tlaplus specs must start with at least four - at each side of \mintinline{text}{MODULE} and four = at the end.
  \item The module name must be the same as filename.
  \item \mintinline{text}{EXTENDS} is the import keyword.
  \item \mintinline{text}{(*...*)} is the block comment. \pluscal spec starts with \mintinline{text}{--algorithm <name>} and ends with \mintinline{text}{end algorithm}.
  \item Initialization of variables.
    \item Where the algorithm is implemented.
  \end{dingautolist}
\end{frame}

\subsection{Values and Standard Operators}

\begin{frame}{Basic values}
  Four basic values in \tlaplus:
  \begin{description}
    \item[String] Must be written en double quotes.
    \item[Integer] Floats are not supported.
    \item[Boolean] Written as TRUE and FALSE.
    \item[Model value] A kind of symbol value.
  \end{description}
\end{frame}

\begin{frame}{Standard operators}
  \begin{table}
    \begin{tabular}{@{} p{2cm}lp{3cm} @{}}
      \toprule
      Operator & Meaning & Example\\
      \midrule
      \mintinline{text}{x = y} & Equals & \mintinline{text}{>> 1 = 2} \newline FALSE \\
      \mintinline{text}{x /= y} & Not Equals & \mintinline{text}{>> 1 /= 2} \newline TRUE\\
      \mintinline{text}{x /\ y} & And & \mintinline{text}{>> TRUE /\ FALSE} \newline FALSE\\
      \mintinline{text}{x \/ y} & Or & \mintinline{text}{>> TRUE \/ FALSE} \newline TRUE\\
      \mintinline{text}{~x} & Not & \mintinline{text}{>> ~TRUE} \newline FALSE\\      
      \mintinline{text}{x := y} & Assignment & PlusCal only\\
      \bottomrule
    \end{tabular}
  \end{table}
\end{frame}

\subsection{Arithmetic Operators}

\begin{frame}{Arithmetic Operators}
  \begin{itemize}
    \item  If you \mintinline{text}{EXTENDS Integers} get the arithmetic operators: \textbf{+}, \textbf{-}, \textbf{\%} and \textbf{*}.
    \item  Decimal division is not supported, only the integer division: \textbf{\mintinline{text}{\div}}.
    \item You also get the range operator \textbf{\mintinline{text}{..}} where \mintinline{text}{a..b} is the set \mintinline{text}{{a, a+1, ..., b-1, b}}.
  \end{itemize}
\end{frame}

\subsection{Complex Types}

\begin{frame}{Complex Types}
  \tlaplus has four complex types:
  \begin{itemize}
    \item Sets.
    \item Tuples or sequences.
    \item Structures.
    \item Functions.
  \end{itemize}
\end{frame}

\subsection{Sets}

\begin{frame}{Sets}
  \begin{exampleblock}{Sets}
    \alert{Unordered} collections of elements of the \alert{same type}.
  \end{exampleblock}

  \mintinline{text}{{1, 2, 42}}

  \mintinline{text}{{{TRUE}, {FALSE, TRUE}, {}}}  
\end{frame}

\begin{frame}{Set Operators}
  \scriptsize
  \begin{table}
    \begin{tabular}{@{} p{2.5cm}p{2cm}p{3.5cm} @{}}
      \toprule
      Operator & Meaning & Example\\
      \midrule
      \mintinline{text}{x \in set} & Is member of & \mintinline{text}{>> 1 \in 1..3} \newline TRUE \\
      \mintinline{text}{x \notin set} & Is not member of & \mintinline{text}{>> 1 \notin 1..2} \newline FALSE\\
      \mintinline{text}{set1 \subseteq set2} & Is subset of & \mintinline{text}{>> {1, 2} \subseteq {1, 2, 3}} \newline TRUE\\
      \mintinline{text}{set1 \union set2} & Union & \mintinline{text}{>> {1, 2} \union (2..3)} \newline \mintinline{text}{{1, 2, 3}}\\
      \mintinline{text}{set1 \intersect set2} & Intersection & \mintinline{text}{>> {1, 2} \intersect (2..3)} \newline \mintinline{text}{{2}}\\
      \mintinline{text}{set1 \ set2} & Difference & \mintinline{text}{>> {1, 2} \ (2..3)} \newline \mintinline{text}{{1}}\\
      \mintinline{text}{Cardinality(set)} & Cardinality \newline (requires EXTENDS FiniteSets) & \mintinline{text}{>> Cardinality({1, 2})} \newline 2\\                  
      \bottomrule
    \end{tabular}
  \end{table}
\end{frame}

\begin{frame}[fragile]{Set Transformations}
  \begin{exampleblock}{Filter sets}
    \mintinline{text}{{x \in set: conditional}}
  \end{exampleblock}

  \begin{minted}{text}
    >> {x \in 1..3: x > 2}
    {3}
  \end{minted}

  \begin{exampleblock}{Map sets}
    \mintinline{text}{{expression: x \in set}}
  \end{exampleblock}

  \begin{minted}{text}
    >> {x * 3: x \in 1..3}
    {3, 6, 9}
  \end{minted}
  
\end{frame}

\subsection{Tuples or Sequences}

\begin{frame}[fragile]{Tuples or Sequences}

  \begin{exampleblock}{Tuples or Sequences}
    \alert{Ordered} collections of elements with the \alert{index starting at 1}.
  \end{exampleblock}

  \begin{minted}{text}
    tuple := <<1, TRUE, {1, 2}>>;
    
    >> tuple[1]
    1
    >> tuple[3]
    {1, 2}
  \end{minted}
\end{frame}

\begin{frame}{Sequence Operators}
  If you \textbf{\mintinline{text}{EXTENDS Sequences}} you get the following additional operators:
  \begin{table}
    \begin{tabular}{@{} p{2.5cm}p{2cm}p{3.5cm} @{}}
      \toprule
      Operator & Meaning & Example\\
      \midrule
      \mintinline{text}{Head(seq)} & Head & \mintinline{text}{>> Head(<<1, 2>>} \newline 1 \\
      \mintinline{text}{Tail(seq)} & Tail & \mintinline{text}{>> Tail(<<1, 2, 3>>} \newline \mintinline{text}{<<2, 3>>} \\
      \mintinline{text}{Append(seq, x)} & Append & \mintinline{text}{>> Append(<<1, 2>>, 3} \newline \mintinline{text}{<<1, 2, 3>>} \\
      \mintinline{text}{seq1 \o seq2} & Combine & \mintinline{text}{>> <<1, 2>> \o <<3>>} \newline \mintinline{text}{<<1, 2, 3>>} \\              \mintinline{text}{Len(seq)} & Length & \mintinline{text}{>> Len(<<1, 2>>)} \newline 2 \\                
      \bottomrule
    \end{tabular}
  \end{table}
\end{frame}

\subsection{Structures or Structs}

\begin{frame}[fragile]{Structures or Structs}

  \begin{exampleblock}{Structures or Structs}
    A \alert{map} of strings to values.
  \end{exampleblock}

  \begin{minted}{text}
    >> [a |-> 1, b |-> <<1, {2, 3}>>].b
       <<1, {2, 3}>>
  \end{minted}
\end{frame}

\subsection{\pluscal Algorithm Body}

\begin{frame}{Assignments}
  Assign an \alert{existing} variable to a value with \mintinline{text}{:=}.

  \begin{alertblock}{Rule of thumb}
    If it's the first time you're using the variable, = is initialization. Every other time, = is equality and := is assignment.
  \end{alertblock}
\end{frame}

\begin{frame}{assert}
  \begin{exampleblock}{assert}
    \mintinline{text}{assert TRUE} does nothing, \mintinline{text}{assert FALSE} raises an error.
  \end{exampleblock}
  In order to use assertions you need to add \mintinline{text}{EXTENDS TLC} to the spec.
\end{frame}

\begin{frame}{skip}
  \begin{itemize}
    \item A \alert{no-op}.
    \item To fill parts of the spec that we haven't filled out yet or conditionals that don't update anything.
  \end{itemize}
\end{frame}

\begin{frame}[fragile]{if}
  \begin{minted}{text}
    if condition1 then
      body1
    elsif condition2 then
      body2
    else
      body3
    end if;
  \end{minted}
\end{frame}

\begin{frame}[fragile]{while loop}
  \begin{minted}{text}
    while condition do
      body
    end while;
  \end{minted}
\end{frame}

\begin{frame}[fragile]{macros}
  To avoid duplications in your specs you can add macros before the begin of the algorithm:
  
  \begin{minted}{text}
    macro name(arg1, arg2) begin
      \* macro's body
    end macro;

    begin
      name(x, y);
    end algorithm;
  \end{minted}

  \begin{alertblock}{Macros limitations}
    You can place assignments, assertions, and if statements in macros, but not while loops. You also cannot assign to any variable more than once.
  \end{alertblock}
\end{frame}


\subsection{Multiple Starting States}

\defverbatim[colored]\multipleStatesSample{
  \inputminted[bgcolor=bgCode]{text}{samples/multiple-states.tla}
}

\begin{frame}[allowframebreaks, fragile]{Multiple Starting States}
  We need a way to specify not just one setup, but an entire space of setups to check our specifications.

  \framebreak
  
  We initialize variables with =, but we can also initialize them with \textbf{\mintinline{text}{\in}}.

  TLC will try running the algorithm with any possible element in the set:

  \multipleStatesSample

  \begin{exampleblock}{BOOLEAN set}
    \tlaplus defines \textbf{BOOLEAN} as the set \mintinline{text}{{TRUE, FALSE}}. This can be useful if you have a variable \mintinline{text}{isReady \in BOOLEAN}.
  \end{exampleblock}
  
  \framebreak

  \begin{exampleblock}{SUBSET}
    \textbf{SUBSET} is the power set, or the set of all subsets.
  \end{exampleblock}

  \begin{minted}{text}
    >> SUBSET 1..2
       {{}, {1}, {2}, {1, 2}}
  \end{minted}

  \framebreak
  
  \begin{exampleblock}{\mintinline{text}{\X}}
    \textbf{\mintinline{text}{set1 \X set2}} is the set of all tuples where the first element is in set1 and the second element in set2.
  \end{exampleblock}

  \begin{minted}{text}
    >> (1..2) \X BOOLEAN
       {<<1, TRUE>>, <<1, FALSE>>, <<2, TRUE>>, <<2, FALSE>>}
  \end{minted}

  \framebreak

  \begin{exampleblock}{[key: set]}
    If \mintinline{text}{x \in [key: set]}, then x is an structure where the value of key is an element in set.
  \end{exampleblock}

  \begin{minted}{text}
    >> [a: (1..2), b: BOOLEAN]
    {[a |-> 1, b |-> TRUE], [a |-> 2, b |-> TRUE],
      [a |-> 1, b |-> FALSE], [a |-> 2, b |-> FALSE]}
  \end{minted}
  
\end{frame}

\defverbatim[colored]\oneInitialStateSample{
  \inputminted[bgcolor=bgCode]{text}{samples/one-initial-state.tla}
}
\defverbatim[colored]\multipleInitialStatesSample{
  \inputminted[bgcolor=bgCode]{text}{samples/multiple-initial-states.tla}
}

\begin{frame}[allowframebreaks]{Beware of state explosion}

  \oneInitialStateSample

  \framebreak

  \multipleInitialStatesSample

  \textbf{1 initial state} vs. $10x10x(2x6)^4 =$ \textbf{2,073,600 initial states}
  
\end{frame}

\subsection{Simulating Nondeterminism}

\begin{frame}[allowframebreaks, fragile]{Simulating Nondeterminism}
  For single process algorithms \pluscal provides two constructs to simulate nondeterminism:

  \framebreak

  \begin{columns}[T,onlytextwidth]
    \column{0.5\textwidth}

    \begin{minted}{text}
    either
      /* branch 1
    or
      /* branch 2
    /* ...
    or
      /* branch n
    end either;
    \end{minted}

  \column{0.5\textwidth}

    \begin{itemize}
      \item TLC will check all branches.
      \item No way to make one branch more likely than others.
      \item If all branches are \emph{macro-valid}, we can place an either inside a macro.
    \end{itemize}
  \end{columns}

  \framebreak

  There are two ways to use the \textbf{\mintinline{text}{with}} statement:
  
  \begin{columns}[T,onlytextwidth]
    \column{0.5\textwidth}

    \begin{minted}{text}
      with var = value do
        /* body
      end with;
    \end{minted}

  \column{0.5\textwidth}

    \begin{minted}{text}
      with var \in set do
        /* body
      end with;
    \end{minted}
    
  \end{columns}

  \begin{itemize}
    \item The former creates a temporary variable, the second is nondeterministic.
    \item TLC will check what happens to all possible assignments of var to elements of set.
    \item with follows \alert{macro rules}: no double assignments and no while loops.
    \item You can place with statements inside macros.
  \end{itemize}
  
\end{frame}

\subsection{Operators}

\defverbatim[colored]\constantsWithOperatorsSample{
  \inputminted[bgcolor=bgCode]{text}{samples/define-constants-with-operators.tla}
}
\defverbatim[colored]\operatorsWithPlusCalVariablesSample{
  \inputminted[bgcolor=bgCode]{text}{samples/operators-pluscal-variables.tla}
}

\begin{frame}[allowframebreaks, fragile]{Operators}
  
  \begin{exampleblock}{Operators}
    An \textbf{operator} is the \tlaplus equivalent to procedures in programming languages.
  \end{exampleblock}

  \begin{minted}{text}
    OpWithArgs(Arg1, Arg2) == Expression

    OpWithoutArgs == Expression
  \end{minted}

  \framebreak

  \begin{exampleblock}{Higher-order Operators}
    Operators that take other opertators as arguments.
  \end{exampleblock}

  \begin{minted}{text}
    Add(a, b) == a + b
    Apply(Op(_, _), x, y) == Op(x, y)

    >> Apply(Add, 2, 3)
    5
  \end{minted}

  \framebreak

  \begin{exampleblock}{Anonymous Operators}
    You can define anonymous operators with \mintinline{tex}{LAMBDA param1, param2, paramN: body}.

    They can only be used as arguments of other operators, not as standalone operators.
  \end{exampleblock}

  \begin{minted}{text}
    >> Apply(LAMBDA x, y: x * y, 2, 3)
    6
  \end{minted}
  
  \framebreak

  \tlaplus also permits definitions of binary (infix) operators.

  For example, the following defines \textbf{$\oplus$} (typed ``(+)'') to mean addition modulo N:

  \begin{minted}{text}
    a (+) b == (a + b) % N
  \end{minted}
  
  There is a table with the user-definable operator symbols at \cite{PlusCal-Manual}.

  \framebreak

  \begin{columns}[T,onlytextwidth]
    \column{0.5\textwidth}
      \centering
      \begin{figure}      
        \includegraphics[width=0.9\textwidth]{figures/user-definable-operator-symbols.png}
        \caption{User-definable operator symbols}
      \end{figure}
    \column{0.5\textwidth}
      \centering
      \begin{figure}
        \includegraphics[width=0.9\textwidth]{figures/ascii-for-typeset-symbols.png}
        \caption{ASCII for typeset symbols}
      \end{figure}
  \end{columns}
  
  \framebreak

  \begin{exampleblock}{Operators as constants}
    If a set of possible values is constant, we define it as an operator instead of a variable. This prevents us from accidentally modifying the set in the algorithm.
  \end{exampleblock}
  
  \constantsWithOperatorsSample

  \begin{exampleblock}{Operators using PlusCal variables}
    If you want to define an operator using the variables of a PlusCal algorithm, you should place it in a \alert{define} block.
  \end{exampleblock}

  Definitions goes below variable definitions and above macro definitions.
  
  \operatorsWithPlusCalVariablesSample
  
\end{frame}


\begin{frame}{Invariants}
  \begin{exampleblock}{Operators as Invariants}
    We can use operators as \textbf{invariants}. Invariants are boolean expressions that are checked at the end of every state of the model execution. If it's ever false, the model fails.
  \end{exampleblock}
\end{frame}


\begin{frame}[allowframebreaks, fragile]{Logical Operators}
  
  \begin{exampleblock}{\mintinline{text}{\A}}
    \textbf{\mintinline{text}{\A}} means ``all elements in a set''. It's used in the form \mintinline{text}{\A x \in set: P(x)}, which means ``for all elements in the set, P(x) is true''.
  \end{exampleblock}

  \begin{minted}{text}
    AllLessThan(set, max) == \A x \in set: x < max

    >> AllLessThan({1, 3}, 4)
    TRUE

    >> AllLessThan({1, 3}, 2)
    FALSE
  \end{minted}

  \framebreak

  \begin{exampleblock}{\mintinline{text}{\E}}
    \textbf{\mintinline{text}{\E}} means ``there exists some element in the set''. It's used in the form \mintinline{text}{\E x \in set: P(x)}, which means ``there is at least one element in the set where P(x) is true''.
  \end{exampleblock}

  \begin{minted}{text}
    SeqOverlapsSet(seq, set) == \E x \in 1..Len(seq): seq[x] \in set

    >> SeqOverlapsSet(<<1, 2, 3>>, {2, 4})
    TRUE

    >> SeqOverlapsSet(<<1, 2, 3>>, {4, 5})
    FALSE
  \end{minted}
  
  \framebreak

  \begin{exampleblock}{\mintinline{text}{=>}}
    \textbf{\mintinline{text}{P => Q}} means that if P is true, then Q is true.
  \end{exampleblock}

  It's equivalent to writing \textbf{\mintinline{text}{~P \/ Q}}.
  
  \framebreak

  \begin{exampleblock}{\mintinline{text}{<=>}}
    \textbf{\mintinline{text}{P <=> Q}} means that either P and Q are both true or P and Q are both false. 
  \end{exampleblock}

  \begin{minted}{text}
    Xor(A, B) == (~A /\ B) \/ (A /\ ~B)
    AlternativeXor(A, B) == ~A <=> B

    >> \A A, B \in BOOLEAN: Xor(A, B) = AlternativeXor(A, B)
    TRUE
  \end{minted}
    
\end{frame}


\subsection{Functions}


\section{\pluscal Koans}

\section{Test the Water}

\section{Workshop of Spec Writing}


\begin{frame}[standout]
  Questions?
\end{frame}

\appendix

\begin{frame}[allowframebreaks]{References}

  \nocite{*}
  \bibliography{PlusCal}
  \bibliographystyle{alpha}

\end{frame}

\end{document}
